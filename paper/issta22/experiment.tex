\section{Experiment}
\label{sec:experiment}

In this section, we present experiments to
evaluate the effectiveness of our proposed evaluation methodology. In
particular, we address the following research questions:

\begin{enumerate}
  \item[\textbf{RQ1}:] How effective is our proposed evaluation model
    for generating diverse test cases? % self bleu score
    
  \item[\textbf{RQ2}:] How effective is our new test case generation
    using \cfg expansion?
  \item[\textbf{RQ3}:] How effective is our proposed evaluation model
    for finding failures given a linguistic capability?
\end{enumerate}

\subsection(Experiment Setup}

For \sa task, input requirements are given for each linguistic
capability. Based on the requirements, \Model generates templates and
lexicons.
\MyPara{Model}
\Bert, \Roberta, \Dbert

\MyPara{Dataset}
reference CFG generation: Treebank corpus
word sentiments: sentiwordnet (https://github.com/aesuli/SentiWordNet)
cfg parser: berkeley neural parser (https://github.com/nikitakit/self-attentive-parser)
synonyms: spacy wordnet synsets (https://spacy.io/universe/project/spacy-wordnet)
seed search dataset: stanford sentiment treebank (sst) dataset (https://nlp.stanford.edu/sentiment/)

\MyPara{Hyper-Parameter}
Number of seed inputs: 10 (random sample)
Number of cfg expansions: 10 (random sample)
Number of synonyms for a word: 10 (random sample)

Besides, we retrain the \sota \sa models aformentioned from test cases
generated from \Chlst and \Model. The goal of the retraining is to
assess comprehensiveness of test cases according to performance of
fine-tuned model. We assume that the comprehensive test cases improve
performance of \sa task when fine-tuning with them. On the other hand,
fine-tuning models with less comprehensive test cases increases
overfitting and decrece model performance.

\MyPara{Dataset}
Retraining dataset: checklist and our proposed model test cases

\MyPara{Hyper-Parameter}
Retraining epoach: 1
Retraining batch_size: 16
