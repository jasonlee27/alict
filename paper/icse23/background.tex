\section{Background}
\label{sec:background}

%% \sw{We should motivate the testing of linguistic capabilities. Give
%%   background of the work that has been done in CheckList: what
%%   capabilities they support and their limitations. Maybe find a
%%   motivating example?}



In this section, we provide a brief background of \Cklst's approach of test case generation
 via an example.
%  Quality of software is verified by ensuring
% the proper working of all functionalities without knowing the internal
% workings of the software. Knowing performances of model on the
% multiple functionalities provides users with better understanding and
% debugging the software. The same principle applies to the NLP model.
% Traditional NLP model evaluation relying on a test set lacks the
% specification of model functionality. However, in NLP domain, there
% are many phenomena on linguistic input such as negation, and
% questionization. Given a NLP task, the phenomena determine
% task-relevant output. Traditional evaluation method neglects them;
% thus, it becomes less efficient to detect and analyze which aspect the
% model yields unexpected outcome.
\Cklst
introduces task-dependent \lcs for monitoring model performance on
each \lc. It assumes that the linguistic phenomena can be represented
in the behavior of the model under test, because each linguistic capability specifies the desired behavior between inputs and outputs. For each \lc, \Cklst creates
test case templates and generates sentences by filling in the values of
each placeholder.

We show an example of \Cklst templates in
Figure~\ref{code:TempEx}. These templates are used for
evaluating a \sa model on the linguistic capability of \SareqExThree.
For the templates defined
from lines~\ref{code:tempex:template:1} to \ref{code:tempex:template:2}, they
have placeholders such as ${it}$, ${air\_noun}$, ${pos\_adj}$. Values
for the placeholders are defined at
line~\ref{code:tempex:airnoun:1},~\ref{code:tempex:posadj:1} and~\ref{code:tempex:itbe}, .
For each template, \Cklst fills in all the combinations of the
values of placeholders, and
the sentences such as ``The flight is good'', ``That
airline was happy'' are generated. These sentences are used for the evaluation
of the \lc.

Despite its simplicity of test case generation, \Cklst has
limitations. First, it relies on manual work for defining template
structure and its placeholder values making the test case
generation a costly process. Moreover, such
manual work generates the limited number of templates and produces high similarity between test cases, and induces bias in the test cases. These limitations motivated the design of our approach.


\begin{figure}[hbp!]
 \centering
 \lstinputlisting[language=python-pretty]{code/checklist_template.py}
 \vspace{-10pt}
 \caption{\CklstTemplateFigCaption}
 \vspace{-10pt}
\end{figure}