\section{Experiment}
\label{sec:experiment}
%

\InputWithSpace{tables/retrain-debug-table}

In this section, we present experiments to evaluate the effectiveness
of our proposed evaluation methodology. In particular, we address the
following research questions (RQs):

\begin{enumerate}[label=\textbf{RQ\arabic*}]
\item \label{rq:one}: Can \tool generate sentences consistent to their
  sentiment label and their target \lc?
\item \label{rq:two}: Can \tool find different bugs from \Cklst?
\item \label{rq:three}: Can the \tool generated testcases have stronger
  generalization ability to test NLP models?
\item \label{rq:four}: Can \tool be useful to find root of bug in the
  NLP models?
%% \item \label{rq:three}: How effective is our new test case generation
%%   using \cfg expansion? % ablation study
\end{enumerate}

To answer the RQs, we need to show: (\romnum{1}) correctness of testcase to
evaluate its target \lc.  (\romnum{2}) effect of testcase distribution on
finding bugs; (\romnum{3}) degree of execution of a NLP model on testcases;
and (\romnum{4}) ability to guide to find cause of bug in a NLP
model.

First, \tool generates sentences and its label by empirical reasoning
derived from \lc statements. Therefore, we meaure acceptibility of the
testcases by verifying the relevancy between sentence and label, and
between sentence and its \lc. The RQ~\ref{rq:one} thereby is answered
by verifying the quality of generated testcases. In this work, we
conduct manual study and measure the correlation of human labeled
testcases and \tool labeled testcases.

Second, evaluation on testcases more equivalent to real world
estimates model performance in practical use more accurately. It means
more various bugs from the input distribution are likely to be
captured.  On the other hand, limited distribution only represents
narrow or uncommon portion of real world, and it leads to detect bugs
within the restricted range. As a result, retraining a model on the
testcases within  obtains lower 

debugs less various bugs.

To answer the RQ~\ref{rq:two}, we retrain
a NLP model on generated testcases and analyze the retraining performance.

it means that retraining model on more
generalized testcases can debug more various bugs found in the
original model.

To answer the RQ~\ref{rq:two}, we retrain a NLP model on generated testcases.

Third, the amount of NLP
model component executed during testing is also critical measurement
for assessing quality of software testing. A high software coverage
results in lower chances of unidentified bugs in the NLP
model. Accordingly, the ability of testcases to produce high portion
of model coverage needs to be measured and answer the
RQ~\ref{rq:three}.

In addition to the detection of bugs in the model,
explaination of the bugs is also important stage of debugging and
repairing the model. Therefore, we anwer RQ~\ref{rq:four} by showing
usefulness of \tool by analyzing the NLP model based on \tool and
finding root of the bug.

\subsection{Experiment Setup}

To resolve these, we conduct practical experiments and show the
effectiveness of \tool. We conduct a maual studyRQ~\ref{rq:one}

we generate test cases and use
them for evaluating model on linguistic capabilities. In this
experiment, We assess the ability to find failures by anlyzing model's
performance on the generated test cases. We also measure the diversity
among the generated test cases using similarites among them. Next, we
answer \ref{rq:three} by retraining \sa model with generated test
cases and measuring performances. The idea behind this is that more
comprehensive inputs becomes closer to real-world distribution and
addresses more type of errors.  Therefore, it leads to improve the
model performance. In this experiment, We retrain the model and
compare performances of the retrained model. Not only that, we conduct
ablation study of \cfg expansion to understand the its impact in our
approach.




%% \MyPara{Seed Input Selection}
%% %
%% For each linguistic capability, we first search all sentences that
%% meet its requirement. Among found sentences, we randomly select 10
%% sentences due to memory constraint.

%% \MyPara{Word Sentiment}
%% %
%% we extract sentiments of words using the
%% \Swn~\cite{baccianella2010sentiwordnet}. The \Swn is a publicly
%% available lexical resource of words on Wordnet with three numerical
%% scores of objectivity, positivity and negativity. Sentiment word
%% labels from the scores are classified from the algorithm from Mihaela
%% \etal~\cite{mihaela2017sentiwordnetlabel}.

%% \MyPara{\Cfg Expansion}
%% %
%% We build a reference \Cfg of natural language from the English Penn
%% \Trb corpora~\cite{mitchell1993treebank,nltkTreebankCorporaWebPage}.
%% The corpus is sampled from 2,499 stories from a tree year \Wsj
%% collection The \Trb provides a parsed text corpus with annotation of
%% syntactic and semantic structure. In this experiment We implement the
%% \trb corpora available through \Nltk, which is a suite of libraries
%% and programs for \Nlp for English. In addition, we parse the seed
%% input using into its CFG using the Berkeley Neural
%% Parser~\cite{kitaev2018constituency, kitaev2019multilingual}, a
%% high-accuracy parser with models for 11 languages. The input is a raw
%% text in natural language and the output is the string representation of
%% parse tree. Next after comparing CFGs between reference and seed input,
%% we randomly select 10 expansions for generating templates due to
%% memory constraint.

%% \MyPara{Synonyms}
%% %
%% \Model searches synonyms of each token from synonym sets extracted
%% from \Wrdnt using \Spacy open-source library for NLP.

%% \MyPara{Models}
%% %
%% We evaluate the following \sa models via \Model:
%% \Bert~\cite{devlin2019bert}, \Roberta~\cite{liu2019roberta} and
%% \Dbert~\cite{sanh2019distilbert}. These models are fine-tuned on \Sstt
%% and their accuracies are \BertAcc, \RobertaAcc and \DbertAcc.

%% \MyPara{Retraining}
%% %
%% We retrain \sa models. we split \Model generated test cases into
%% train/validation/test sets with the ratio of 8:1:1. The number of
%% epochs and batch size for retraining are 1 and 16 respectively.
