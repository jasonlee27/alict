\section{Experimental Setup}
\label{sec:experiment}
%

% \InputWithSpace{tables/retrain-debug-table}

In this section, we present the setup of the experiments to evaluate
the effectiveness of \tool{}. We address the following research
questions (RQs):

\sw{We may miss a RQ for the test results using \tool. In the setup,
  we have not said which sentiment analysis models we tested, and how
  we measure the results (e.g., number of misclassified test cases).}

\begin{enumerate}[label=\textbf{RQ\arabic*}]
\item \label{rq:one}: Can \tool generate consistent test sentence and
  its oracle?
\item \label{rq:two}: Is \tool generated testcases relevant to be used
  for their \lc evaluation?
\item \label{rq:three}: Can \tool generate more diverse test cases
  than \Cklst?
\item \label{rq:four}: Can \tool be useful to find root causes of bugs
  in the \sa models?
%% \item \label{rq:three}: How effective is our new test case generation
%%   using \cfg expansion? % ablation study
\end{enumerate}

%To answer the RQs, we need to show: (\romnum{1}) correctness of testcase to
%evaluate its target \lc.  (\romnum{2}) effect of testcase distribution on
%finding bugs; (\romnum{3}) degree of execution of a \sa model on testcases;
%and (\romnum{4}) ability to guide to find cause of bug in a \sa
%model.

\sw{Make clear (earlier in the paper): a test case is a sentence in a
  linguistic capability with a sentiment label.}

\paragraph{\textbf{RQ1 and RQ2}} As described in Section \ref{sec:approach},
\tool generates test cases in two steps: specification-based seed
generation and syntax-based sentence expansion. These automated steps
may generate seed/expanded sentences marked with incorrect sentiment
labels or categorized into wrong linguistic capabilities. For example,
the search rule and template defined in a linguistic capability may
not always generate seed sentences in that capability or with the
correct label.  To answer RQ1 and RQ2, we perform a manual study to
measure the correctness of the sentiment labels and linguistic
capabilities associated with the seed/expanded sentences, produced by
\tool.

In the manual study, we randomly sample three sets of pairs of seed
sentences and corresponding \lc from seed test cases. For each set, we
also select expanded sentences that \tool generated from the formerly
sampled seed sentences. In this experiment, each set has 100 sentences
(50 from seed sentences and 50 from expanded sentences) and 300
sentences, in total, are used for the manual study. For each sampled
set, two subjects are provided with the same set of sampled
sentences. The subjects are asked for scoring the two following:
\textbf{1. relevancy score between sentence and its associated \lc}:
this score measures the amount of appropriateness of the use of
sentence for evaluating the model on its \lc.  The scores are discrete
ranging from 1 to 5, and each represents ``strongly not relevant'' to
``strongly relevant'' respectively. \textbf{2. sentiment score of
  sentence}: this score measures the level of sentence sentiment. It
is also discrete, and it ranges from 1 to 5 representing ``strongly
negative'' to ``strongly positive'' respectively. In this work, we
collect manual study scores from 6 subjects in total. From the
collected scores, we measure the following metrics:

\begin{eqnarray}
  sentiment\_relecancy &= \sum_{i} \delta(label_{S^2LCT}!=label_{human}) \label{metric:srel} \\
  LC\_relevancy_{AVG} &= \frac{1}{\#data}\cdot\sum_{i} LC\_relevancy_i \label{metric:lcrel}
\end{eqnarray}

The equation~\ref{metric:srel} represent the number of test cases that
their labels assigned from are different between \tool and
human. Higher number of this metric indicates worse correlation of
test oracle that \tool generated with human. In addition, the
equation~\ref{metric:lcrel} represents the average score of the
relevancy score between sentence and its associated \lc. higer average
score means that higher human-level agreement of the use of sentence
for its \lc, resulting in higher suitability of the use of the
testcases for evaluating model on the \lc. Given the metrics, we
answer RQ1 and RQ2 by the metrics from the equation~\ref{metric:srel}
and equation~\ref{metric:lcrel} respectively, thereby, show its ability
of \tool to understand human intelligence.

\paragraph{RQ3.}
Recall that a key limitation of \Cklst is that its template-based
approach that relies on significant manual efforts may not generate
test cases that comprehensively cover the sentences in a linguistic
capability. \tool, instead, automatically generates test cases based
on a search dataset and the syntax in a large reference corpus. We
expect \tool can generate a more diverse test suite than \Cklst. To
measure diversity, we follow the approach presented by Ma et
al. \cite{ma2018deepgauge}, where the authors measure the coverage of
NLP model intermediate states as corner-case neurons.  \sw{Why is this
  a good metric for diversity?}  Specifically, we use the \sw{are
  these metrics we define or are they from ma2018deepgauge?} two
coverage metrics in existing work \cite{ma2018deepgauge},
\textit{boundary coverage} (BoundCov) and \textit{strong activation
  coverage} (SActCov), as our metrics to evaluate the test suite
diversity.

\begin{equation}
\begin{split}
    \text{UpperCornerNeuron}(\mathcal{X}) = \{n \in N | \exists x \in \mathcal{X}: f_n(x) \in (high_n, +\infty)\}; \\
    \text{LowerCornerNeuron}(\mathcal{X}) = \{n \in N | \exists x \in \mathcal{X}: f_n(x) \in (-\infty, low_n)\}; \\
\end{split}
    \label{eq:corner}
\end{equation}

\noindent Eq. \ref{eq:corner} shows the formal definition of the corner-case neuron of the NLP model $f(\cdot)$, where $\mathcal{X}$ is the given test suite, $N$ is the number of neurons in model $f(\cdot)$, $f_n(\cdot)$ is the $n^{th}$ neuron's output, and $high_n, low_n$ are the $n^{th}$ neurons' output bounds on the model training dataset.
Eq. \ref{eq:corner} can be interpreted as the collection of neurons that emit outputs beyond the model's numerical boundary.

\begin{equation}
\begin{split}
     & BoundCov(\mathcal{X}) = \frac{|UpperCornerNeuron(\mathcal{X})| + |LowerCornerNeuron| }{2 \times |N|} \\ 
     & SActCov(\mathcal{X}) = \frac{|UpperCornerNeuron(\mathcal{X})|} {|N|} \\ 
\end{split}
    \label{eq:coverage}
\end{equation}

\noindent The formal definition of our coverage metrics are shown in Eq.\ref{eq:coverage}, where BoundCov measures the coverage of neurons that produce outputs that exceed the upper or lower bounds, and SActCov measures the coverage of neurons that create outputs that exceed the lower bound.
Higher coverage indicates the test suite is better for triggering the corner-case neurons, thus better test suite diversity.



% ($Cov(\mathcal{X})$ in \equref{eq:coverage}),
% where $N$ is the total number blocks, $\mathbb I(\cdot)$ is the indicator function, and
% $(B_i(x) > \tau_i))$ represents whether $i^{th}$ block is activated by input $x$~(the definition of $B_i$ and $\tau_i$ are the same with \equref{eq:new2} and \equref{eq:new3}).
% Because AdNNs activate different blocks for decision making, then a higher block coverage indicates the test samples cover more decision behaviors. 


To answer {\bf RQ3}, for each NLP model under test, we first feed its training dataset to compute each neuron's lower and upper bounds. After that, we randomly select 100 \sw{Why 100, not all?} test cases from \tool and \Cklst as the test suite and compute the corresponding coverage metrics. 
\cm{we repeat this process and record both the average and variance value of each coverage}
% For each subject, we randomly select 100 seed samples from the test dataset as seed inputs. We then feed the same seed inputs into \tool and \texttt{ILFO} to generate test samples. 
% Finally, we feed the generated test samples to AdNNs and measure block coverage.
% We repeat this process 10 times and record the average coverage and the variance.
% The results are shown in \tabref{tab:coverage} last two columns. 

%
%The amount of \sa model
%components \sw{?} executed during testing is a critical measurement for
%assessing quality of software testing. A high software coverage
%results in higher chances of unidentified bugs in the \sa model. On
%the other hand, limited distribution only represents narrow portion of
%real world covering limited execution behaviors in a \sa model. It leads
%to detect bugs within the restricted execution behaviors.  Therefore,
%test cases more representative of real-world data result in more generalized
%distribution and higher coverage of the \sa model.
%Therefore, we answer the {\bf RQ2} by
%measuring the neural coverage of the \sa model.
%
%Specifically, we implemented DeepXplore to measure the \sa model
%coverage~\cite{pei2017deepxplore}. \Dxp is the first efficient
%white-box testing framework for large-scale \dl systems. It introduces
%neuron coverage of a set of test inputs as the ratio of the number of
%unique activated neurons and the total number of neurons in input \dl
%system. In this experiment, we compute the neuron coverage of a test
%cases from \tool and \Cklst on the fine-tuned \sw{first time mention fine-tuning: reader does not know how we fine-tune} \sa model of
%\bertsamodel and compare the coverage between \tool and \Cklst.

\paragraph{RQ4.} To demonstrate that \tool can help developers to understand the bugs in the sentiment analysis modes, we conduct experiments to visualize 




\sw{@Simin: add the setup of the bug explanation case study here.}
%In addition to the detection of bugs \sw{have we defined ``bug" in this context?} in the model,
%explanation of the bugs is also important for debugging and
%repairing the model. Therefore, we answer {\bf RQ3} by analyzing the \sa model based on \tool test cases to find the root causes of the bugs.
%
%Specifically, we adapt \Denas{} for this experiment. \sw{Say more what DENAS does.} Rules generated from
%\Denas{} are interpretable functions mapping certain features of the
%input to the expected output of a deep \nn system, and the generated
%rules are considered as the behaviors of the deep \nn system. To
%analyze the bug of a \sa model, we generates rules over the test
%inputs using \Denas{} and identifies faulty rules for the failed test
%inputs.
%\sw{Previous sounds vague. Any specific adaptation of DENAS we did?}
%We manually identify the root of the faulty rules and find the
%root causes of bug in the end.








\noindent\textbf{Implementation Details.}









\sw{Missing: environment running these experiments.}

%% \MyPara{Seed Input Selection}
%% %
%% For each linguistic capability, we first search all sentences that
%% meet its requirement. Among found sentences, we randomly select 10
%% sentences due to memory constraint.

%% \MyPara{Word Sentiment}
%% %
%% we extract sentiments of words using the
%% \Swn~\cite{baccianella2010sentiwordnet}. The \Swn is a publicly
%% available lexical resource of words on Wordnet with three numerical
%% scores of objectivity, positivity and negativity. Sentiment word
%% labels from the scores are classified from the algorithm from Mihaela
%% \etal~\cite{mihaela2017sentiwordnetlabel}.

%% \MyPara{\Cfg Expansion}
%% %
%% We build a reference \Cfg of natural language from the English Penn
%% \Trb corpora~\cite{mitchell1993treebank,nltkTreebankCorporaWebPage}.
%% The corpus is sampled from 2,499 stories from a tree year \Wsj
%% collection The \Trb provides a parsed text corpus with annotation of
%% syntactic and semantic structure. In this experiment We implement the
%% \trb corpora available through \Nltk, which is a suite of libraries
%% and programs for \Nlp for English. In addition, we parse the seed
%% input using into its CFG using the Berkeley Neural
%% Parser~\cite{kitaev2018constituency, kitaev2019multilingual}, a
%% high-accuracy parser with models for 11 languages. The input is a raw
%% text in natural language and the output is the string representation of
%% parse tree. Next after comparing CFGs between reference and seed input,
%% we randomly select 10 expansions for generating templates due to
%% memory constraint.

%% \MyPara{Synonyms}
%% %
%% \Model searches synonyms of each token from synonym sets extracted
%% from \Wrdnt using \Spacy open-source library for NLP.

%% \MyPara{Models}
%% %
%% We evaluate the following \sa models via \Model:
%% \Bert~\cite{devlin2019bert}, \Roberta~\cite{liu2019roberta} and
%% \Dbert~\cite{sanh2019distilbert}. These models are fine-tuned on \Sstt
%% and their accuracies are \BertAcc, \RobertaAcc and \DbertAcc.

%% \MyPara{Retraining}
%% %
%% We retrain \sa models. we split \Model generated test cases into
%% train/validation/test sets with the ratio of 8:1:1. The number of
%% epochs and batch size for retraining are 1 and 16 respectively.




%
\begin{figure}
    \centering
    \includegraphics[width=0.5\textwidth]{figs/covergae.pdf}
    \caption{The coverage results of the generated test
      samples. \sw{Make the figures larger.}}
    \label{fig:coverage}
\end{figure}


Fig. \ref{fig:coverage} shows the coverage results of the generated
test samples, where the red line represents \tool and the black line
represents \Cklst.  Each column in Fig. \ref{fig:coverage} represent
the results for one NLP model, the first row is the \textit{BoundCov}
results and the second row is the \textit{SActCov} results.  From the
results, we make two observations observations. First, for \emph{all}
experimental settings (\eg NLP model, coverage metric), \tool achieves
high coverage than \Cklst. Recall that a higher coverage implies the
test case in the test suite is more diverse and rarely to have a
statistical distribution similar to the model training data. As a
result, a test suite with greater coverage complements the model
training data distribution (\ie hold-out testing data) better.  The
experimental results confirm that \tool can generate more diverse test
cases to complement the hold-out testing data for testing NLP models.
\sw{What does the growth trend in each figure indicate?} \sw{Does the
  absolute numbers or relative difference of the two lines on y axis
  mean anything concrete? E.g., how significant is the improvement in
  diversity?}

Another interesting finding is that for each NLP model, there is no
fixed relationship between \textit{BoundCov} and \textit{SActCov}. In
other words, while a test suite may produce higher \textit{BoundCov}
for some models, the same test suite may get higher \textit{SActCov}
for other NLP models.  Recall that \textit{BoundCov} measures both the
upper and lower corner neurons and \textit{SActCov} measures only the
upper corner neurons.  Such observation implies that the upper and
lower corner neurons are distributed unevenly, and measuring only one
of them is not enough.


%\input{CM/explain}
