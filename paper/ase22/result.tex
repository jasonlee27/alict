\section{Experimental Results}
\label{sec:result}

This section presents experiment results and answer the RQs by
studying the results quantatively and qualitatively.

\InputWithSpace{tables/manual-study-table}
\InputWithSpace{tables/test-results-table}
\subsection{RQ1: \tool Sentiment Label Consistency}
Table~\ref{table:ManualStudy} shows results of our manual study. The
first column represents type of test case. The number of test cases
used for the study is represented in second column. The label
consistency score defined in equation~\ref{metric:srel} is shown in
column 3.

We observe that \emph{\tool generates test cases that consistently
  label their sentiment correctly.}  Column 3 shows that the label
consistency scores are 0.83 and 0.84 for the seed and expanded
sentences, respectively. We can observe that the scores are not same
as maximum scores meaning that there are inconsistency on labels
between \tool and subjects. The cuases of inconsistency in the scores
are two following: First, complicated sentence leads subjects to
misunderstand its meaning. Second, phrase in a sentence introduces its
multiple interpretations to understand its sentiment. For example, the
word ``easy'' could be interpreted as both compliment and back-handed
insult.  This means that \tool generates test oracles consistent with
human understanding most of the time. It is observed that there is
little difference of the scores between the seed and expanded
sentences. This implies that the syntax-based sentence expansion in
\tool preserves the sentiment as its seed.
%% \sw{What are
%%   the causes of inconsistency? Looks like the main source of
%%   inconsistency comes from seed generation? Any insights on how it
%%   happened (e.g., issue with original labels, search rules, or
%%   templates?}

\subsection{RQ2: Correctness of Linguistic Capability Categorization}
The \lc relevancy score defined in~\ref{metric:lcrel} is shown in
column 4 of Table \ref{table:ManualStudy}. The result shows that
\emph{\tool generates test cases that are correctly categorized to the
  corresponding linguistic capabilities most of the time.}  The \lc
relevancy scores for the seed and expanded sentences are both 0.9,
achieving high order of agreement with human assessment. The fact that
the expanded sentences generated by \tool also have same level of \lc
relevancy as the seed sentences shows that the syntax-based sentence
expansion retains the linguistic capabilities.
%% \sw{Source of incorrect
%% categorization of linguistic capabilities?}


\subsection{RQ3: Test Suite Diversity}

We first show evaluation results of three \sa models via
table~\ref{table:TestResult}. In the table, first column lists
linguistic capability for \sa task, and Columns 2-3 show the number of
seed and expanded test cases respectively. Columns 4-5 show the
failure rate in percentage by evaluating the \sa models on the seed
and expanded test cases respectively. Column 6 shows the number of
expanded test cases failed, but their seed test cases passed.  From
the table, we can observe that LC3 has no expanded sentences from its
seeds. It means that the 50 seeds selected in this experiment does not
provide any validated expansions. It is because the randomness of
selection of 50 seeds determines the number of validated
expansions. we expect that more selection of seeds increases the
probability of availability of their expansion. In addition, LC1 and
LC5 obtains 19 and 26 seeds respectively. It represents that \tool
finds and generates their seeds lower than the target number of seeds,
which is 50 in this experiment. In this case, use of larger search
dataset increases the amount of seeds for the \lcs. From the table, it
is found that all three models achieves better performance on LC9 and
LC2 while they obatains high failure rates on other \lcs. Finally, we
find that there are number of a number of failed expanded test cases
succeeded before the expansion (pass on seed, but fail on its expanded
test cases). This shows that phase of syntax-based sentence expansion
in \tool captures failure-inducing expansion given a seed. It finally
shows usefulness of \tool since the different test results between
seed and expanded cases provide accurate guidance for debugging model.


\begin{figure}
    \centering
    \includegraphics[width=0.5\textwidth]{figs/covergae.pdf}
    \caption{The coverage results of the generated test
      samples. \sw{Make the figures larger.}}
    \label{fig:coverage}
\end{figure}


Fig. \ref{fig:coverage} shows the coverage results of the generated
test samples, where the red line represents \tool and the black line
represents \Cklst.  Each column in Fig. \ref{fig:coverage} represent
the results for one NLP model, the first row is the \textit{BoundCov}
results and the second row is the \textit{SActCov} results.  From the
results, we make two observations observations. First, for \emph{all}
experimental settings (\eg NLP model, coverage metric), \tool achieves
high coverage than \Cklst. Recall that a higher coverage implies the
test case in the test suite is more diverse and rarely to have a
statistical distribution similar to the model training data. As a
result, a test suite with greater coverage complements the model
training data distribution (\ie hold-out testing data) better.  The
experimental results confirm that \tool can generate more diverse test
cases to complement the hold-out testing data for testing NLP models.
\sw{What does the growth trend in each figure indicate?} \sw{Does the
  absolute numbers or relative difference of the two lines on y axis
  mean anything concrete? E.g., how significant is the improvement in
  diversity?}

Another interesting finding is that for each NLP model, there is no
fixed relationship between \textit{BoundCov} and \textit{SActCov}. In
other words, while a test suite may produce higher \textit{BoundCov}
for some models, the same test suite may get higher \textit{SActCov}
for other NLP models.  Recall that \textit{BoundCov} measures both the
upper and lower corner neurons and \textit{SActCov} measures only the
upper corner neurons.  Such observation implies that the upper and
lower corner neurons are distributed unevenly, and measuring only one
of them is not enough.


\subsection{RQ4: Use \tool for Debugging}
