\section{Experimental Results}
\label{sec:result}

This section presents experiment results and answer the RQs by
studying the results quantatively and qualitatively.

\InputWithSpace{tables/manual-study-table}
\InputWithSpace{tables/test-results-table}

\subsection{RQ1: \tool Sentiment Label Consistency}
Table~\ref{table:ManualStudy} shows the statistics of scores collected
from the manual study of \tool. First column represents type of test
case. The number of test cases used for the study is represented in
second column. Label consistency score and \lc relevancy score defined
at equation~\ref{metric:srel} and~\ref{metric:lcrel} are shown at the
remaining columns. First, it is shown that high label consistency
score over 0.8. It means that \tool generates test oracles consistent
to human understanding. It is also observed that there is no
difference of the scores between seed and expanded sentences. The
observation implies that the expansion method in \tool preserves
sentiment of expanded sentences as its seed, and provides its
reliability. These observations answer RQ1 and conclude that \tool
generates test case with high consistency between test sentence and
its oracle.

\subsection{RQ2: Correctness of Linguistic Capability Categorization}
The result in table~\ref{table:ManualStudy} also shows that \tool
achieves high order of agreement with human assessment on \lc
relevancy. In parallel, it is observed that the expanded sentences
generated from \tool also have same level of \lc
relevancy. Accordingly, we answer the RQ2 by concluding that
\tool generates \lc relevant test cases with the agreement with human.

\jl{The following needs to be in RQ2 result section after simin
  completes RQ2 result section} Table~\ref{table:TestResult} shows
results of evaluation of three \sa models. First column lists
linguistic capability for \sa task, and Columns 2-3 show the number of
seed and expanded test cases respectively. Columns 4-5 show the
failure rate in percentage by evaluating the \sa models on the seed
and expanded test cases respectively. Column 6 shows the number of
expanded test cases failed, but their seed test cases passed.  From
the table, we can observe that LC3 has no expanded sentences from its
seeds. It means that the 50 seeds selected in this experiment does not
provide any validated expansions. It is because the randomness of
selection of 50 seeds determines the number of validated
expansions. we expect that more selection of seeds increases the
probability of availability of their expansion. In addition, LC1 and
LC5 obtains 19 and 26 seeds respectively. It represents that \tool
finds and generates their seeds lower than the target number of seeds,
which is 50 in this experiment. In this case, use of larger search
dataset increases the amount of seeds for the \lcs. Finally, we find
that there are number of a number of failed expanded test cases succeeded
before the expansion (pass on seed, but fail on its expanded test
cases). This shows that phase of syntax-based sentence expansion in
\tool captures failure-inducing expansion given a seed. It finally
shows usefulness of \tool since the different test results between seed
and expanded cases provide accurate guidance for debugging model.


\subsection{RQ3: Test Suite Diversity}

\begin{figure}
    \centering
    \includegraphics[width=0.5\textwidth]{figs/covergae.pdf}
    \caption{The coverage results of the generated test
      samples. \sw{Make the figures larger.}}
    \label{fig:coverage}
\end{figure}


Fig. \ref{fig:coverage} shows the coverage results of the generated
test samples, where the red line represents \tool and the black line
represents \Cklst.  Each column in Fig. \ref{fig:coverage} represent
the results for one NLP model, the first row is the \textit{BoundCov}
results and the second row is the \textit{SActCov} results.  From the
results, we make two observations observations. First, for \emph{all}
experimental settings (\eg NLP model, coverage metric), \tool achieves
high coverage than \Cklst. Recall that a higher coverage implies the
test case in the test suite is more diverse and rarely to have a
statistical distribution similar to the model training data. As a
result, a test suite with greater coverage complements the model
training data distribution (\ie hold-out testing data) better.  The
experimental results confirm that \tool can generate more diverse test
cases to complement the hold-out testing data for testing NLP models.
\sw{What does the growth trend in each figure indicate?} \sw{Does the
  absolute numbers or relative difference of the two lines on y axis
  mean anything concrete? E.g., how significant is the improvement in
  diversity?}

Another interesting finding is that for each NLP model, there is no
fixed relationship between \textit{BoundCov} and \textit{SActCov}. In
other words, while a test suite may produce higher \textit{BoundCov}
for some models, the same test suite may get higher \textit{SActCov}
for other NLP models.  Recall that \textit{BoundCov} measures both the
upper and lower corner neurons and \textit{SActCov} measures only the
upper corner neurons.  Such observation implies that the upper and
lower corner neurons are distributed unevenly, and measuring only one
of them is not enough.


\subsection{RQ4: Use \tool for Debugging}
