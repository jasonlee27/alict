%%% defs-code.tex
%%% Utility Commands for Code Listings
%%% Version 1.0.2

%%%==========
%%% COPIED FROM vkuncak/doc/vmcai09/defs.tex
\definecolor{gray}{RGB}{211,211,211}
\newcommand{\jbasicstyle}{\normalfont\sffamily} % Style of code
\newcommand{\textcode}[1]{{#1}}
\newcommand{\jnumberstyle}{\scriptsize}
\newcommand{\Hilight}{\makebox[0pt][l]{\color{gray}\rule[-3pt]{0.80\linewidth}{9pt}}}

\colorlet{punct}{red!60!black}
\definecolor{background}{HTML}{EEEEEE}
\definecolor{delim}{RGB}{20,105,176}
\colorlet{numb}{magenta!60!black}

\makeatletter
\newenvironment{btHighlight}[1][]
{\begingroup\tikzset{bt@Highlight@par/.style={#1}}\begin{lrbox}{\@tempboxa}}
{\end{lrbox}\bt@HL@box[bt@Highlight@par]{\@tempboxa}\endgroup}

\newcommand\btHL[1][]{%
  \begin{btHighlight}[#1]\bgroup\aftergroup\bt@HL@endenv%
}
\def\bt@HL@endenv{%
  \end{btHighlight}%
  \egroup
}
\newcommand{\bt@HL@box}[2][]{%
  \tikz[#1]{%
    \pgfpathrectangle{\pgfpoint{1pt}{0pt}}{\pgfpoint{\wd #2}{\ht #2}}%
    \pgfusepath{use as bounding box}%
    \node[anchor=base west, fill=orange!30,outer sep=0pt,inner xsep=1pt, inner ysep=0pt, rounded corners=1pt, minimum height=\ht\strutbox,#1]{\raisebox{1pt}{\strut}\strut\usebox{#2}};
  }%
}

\newenvironment{btHighlightLine}[1][]
{\begingroup\tikzset{bt@HighlightLine@par/.style={#1}}\begin{lrbox}{\@tempboxa}}
{\end{lrbox}\bt@HLLine@box[bt@HighlightLine@par]{\@tempboxa}\endgroup}

\newcommand\btHLLine[1][]{%
  \begin{btHighlightLine}[#1]\bgroup\aftergroup\bt@HLLine@endenv%
}
\def\bt@HLLine@endenv{%
  \end{btHighlightLine}%
  \egroup
}
\newcommand{\bt@HLLine@box}[2][]{%
  \tikz[#1]{%
    \pgfpathrectangle{\pgfpoint{0pt}{-1pt}}{\pgfpoint{\wd #2}{\ht #2}}%
    \pgfusepath{use as bounding box}%
    \node[anchor=base west, fill=orange!30,outer sep=0pt,inner xsep=0pt, inner ysep=0pt, minimum height=\ht\strutbox+3pt, minimum width=4.1cm,#1] (line-bg) {};
    \node[right = 0 of line-bg.west, outer sep=0pt, inner xsep=0pt, inner ysep=0pt]{\raisebox{0pt}{\strut}\strut\usebox{#2}};
  }%
}
\makeatother

% Disable tikz's error message complaining redefining dollor-sign.  WARNING: this may indeed cause problems but hopefully not
\makeatletter
\global\let\tikz@ensure@dollar@catcode=\relax
\makeatother


\lstdefinelanguage{pseudo}
{
  morekeywords={},
  keywordstyle=\bfseries,
  lineskip=-0.1em,
  numbers=left, % none for no numbers
  numberstyle=\jnumberstyle,
  numbersep=4pt,
  basicstyle=\jbasicstyle,
  breaklines=true,
  breakautoindent=true,
  tabsize=2,
  columns=fullflexible,
  morecomment=*[l][\textsl]{//},
  mathescape=true,
  xleftmargin=10pt,
%  mathescape=false,
}

\lstdefinelanguage{todo-comment}
{
  morekeywords={},
  keywordstyle=\bfseries,
  lineskip=-0.1em,
  numbers=none,
  basicstyle=\jbasicstyle,
  breaklines=true,
  breakautoindent=true,
  tabsize=2,
  columns=fullflexible,
  morecomment=*[l][\textsl]{//},
  mathescape=true,
  xleftmargin=-10pt,
%  mathescape=false,
}

\lstdefinelanguage{json-pretty}
{
  basicstyle=\normalfont\ttfamily,
  numbers=none,
  stepnumber=1,
  numbersep=8pt,
  showstringspaces=true,
  breaklines=true,
  %% frame=single,
  %% backgroundcolor=\color{gray},
  literate=
    *{0}{{{\color{numb}0}}}{1}
     {1}{{{\color{numb}1}}}{1}
     {2}{{{\color{numb}2}}}{1}
     {3}{{{\color{numb}3}}}{1}
     {4}{{{\color{numb}4}}}{1}
     {5}{{{\color{numb}5}}}{1}
     {6}{{{\color{numb}6}}}{1}
     {7}{{{\color{numb}7}}}{1}
     {8}{{{\color{numb}8}}}{1}
     {9}{{{\color{numb}9}}}{1}
     {:}{{{\color{punct}{:}}}}{1}
     {,}{{{\color{punct}{,}}}}{1}
     {\{}{{{\color{delim}{\{}}}}{1}
     {\}}{{{\color{delim}{\}}}}}{1}
     {[}{{{\color{delim}{[}}}}{1}
     {]}{{{\color{delim}{]}}}}{1},
}

\lstdefinelanguage{python-pretty}
{
  language=python,
  numbers=none,
  basicstyle=\jbasicstyle,
  numberstyle=\jnumberstyle,
  breaklines=true,
  columns=fullflexible,
  xleftmargin=16pt,
  showstringspaces=false,
  keywordstyle=\bfseries\color{green!40!black},
  commentstyle=\itshape\color{purple!40!black},
  identifierstyle=\color{blue}
}

\lstset{escapeinside={(*@}{@*)}}

\newcommand{\JsonIn}[1]{{\lstinline[language=json-pretty, basicstyle=\small\ttfamily]@#1@}}
\newcommand{\PythonIn}[1]{{\lstinline[language=python-pretty, basicstyle=\small\ttfamily]@#1@}}

%%%==========
