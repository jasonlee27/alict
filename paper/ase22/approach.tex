\section{Specification- and Syntax-based Linguistic Capability Testing}
%% \label{sec:approach}

\begin{figure*}
  \centering
  \includegraphics[width=\linewidth]{figs/overview.pdf}
  \caption{Overview of \tool{}.}
  \label{fig:overview}
\end{figure*}

\sw{Some paragraphs in this overview part should go earlier in the
  paper. I am just writing them here for now as I don't think we have
  them in the intro/background yet.}  We design and implement
\emph{Specification- and Syntax-based Linguistic Capability Testing
  (\tool{})} to automatically generate test cases to test the
robustness of sentiment analysis models. We identify four goals for a
large and effective test suite:

\begin{description}
\item[{\bf G1}] the test suite should contain realistic sentences;
\item[{\bf G2}] the test suite should cover diverse syntactic
  structures;
  \item[{\bf G3}] each test case should be categorized into a
    linguistic capability;
\item[{\bf G4}] the label of each test case should be
automatically and accurately defined.
\end{description}

\Chlst's templates generate complete and realistic sentences, and each
template maps to a linguistic capability, satisfying {\bf G1} and {\bf
  G3}. But \Chlst only uses \sw{X} manually created templates to
generate its test suite; all test cases generated by the same template
share the same syntactic structure, thus violating {\bf G2}. In
addition, the label of each \Chlst test case has to be decided
manually, associated with each template, violating {\bf G4}.

We present \tool{}, a new linguistic capability test case generation
tool, that satisfies all of these criteria.  \sw{I could not summarize
  a cohesive idea that drives our design. Leaving it here to fill in.
  Also, I felt my writing below still lacks justification for some
  design choices (e.g., why we do the differentiation).}  Figure
\ref{fig:overview} shows the overview of \tool{}, which consists of
two phases.  The \emph{specification-based seed generation} phase
performs rule-based searches from a real-world dataset ({\bf G1}) and
template-based transformation to obtain the initial seed sentences.
The search rules (e.g., search for neutral sentences that do not
include any positive or negative words) and transformation templates
(e.g., \sw{add an example}) are defined in the \emph{linguistic
  capability specifications}, which guarantee that each resulting seed
conforms to a specific linguistic capability ({\bf G3}) and is
labelled correctly ({\bf G4}).

The \emph{syntax-based sentence expansion} phase expands the seed
sentences with additional syntactic elements (i.e., words \sw{more?})
to cover many real-world syntactic structures ({\bf G2}). It first
performs a syntax analysis to identify the part-of-speech (PoS) tags
that can be inserted to each seed, by comparing the PoS parse trees
between the seed sentence and many other sentences from a large
reference dataset. Each identified tag is inserted into the seed as a
\emph{mask}. It then uses an NLP recommendation model (i.e., BERT
\cite{}) to suggest possible words. If a resulting sentence is
validated to be consistent with the specification which additionally
defines the rules for expansion (e.g., the expanded word should be
neutral), {\bf G3} and {\bf G4} are still satisfied.  Last, because
some validated sentences may include unrealistic suggested words or
use a rare syntactic structure \sw{is this the right motivation to do
  selection?}, we use a heuristic (i.e., combination of the confidence
score from the NLP recommendation model and the frequency of the
syntactic productions) to select the more realistic and frequent
expanded sentences into \tool{}'s test suite.

We now describe each phase of \tool{} in detail.

%\Model generates input \sents with the following phases illustrated
%in \ref{fig:OverallModel}: 1. search phase searches seed \sents
%according to its \req of \lc, 2. seed parsing phase parses the found
%seed \sents and extract their \cfg, 3. reference parsing phase
%collects \pcfg from large corpus, 4. production differentiation phase
%identifies structural expansion candidates for input expansion, and
%5.\sent selection phase generates natural expanded \sent. In this
%section, we provide more details on each phase.

\subsection{Specification-based Seed Generation}
The seed generation phase of \tool starts by searching sentences in a
real-world dataset that match the rules defined in the linguistic
capability specification, and then transforming the matched sentences
using templates to generate seed sentences that conform to individual
linguistic capabilities. The reasons for this design choice are
twofold.  First, while generally judging which linguistic capability
any sentence falls into and which label it should have is infeasible,
there exist simple rules and templates to allow classifying the
resulting sentences into individual linguistic capabilities and with
the correct labels, with high confidence.  This enables us to test
each linguistic capability individually.  Second, searching from a
real-world dataset ensures that the sentences used as test cases for
testing linguistic capabilities are realistic and diverse. The diverse
test cases are more likely to achieve a high coverage of the target
model's functionality in each linguistic capability, thus detecting
more errors.

%The search phase in \Model searches inputs in dataset and selects
%subset of input \sents in the dataset that meets the \lc
%\req. The idea behind this phase is that input distribution of
%\lc is important to generate inputs relevant to \lc. \Lc explains
%expected behaviors of NLP model on specific types of input and
%output. The NLP model is evaluated on how much it performs on the
%input and output. Thus, \lc introduces the constraints of the input
%data. Input data from the constrained distribution are only qualified
%to be used for evaluating the NLP model on the \lc.  In addition,
%diversity in inputs is important to evaluate NLP models on the
%\lc. Inputs that differ are more likely to cover the NLP model
%behavior, and more coverage increases trustworthiness of the
%evaluation. To generate inputs from same distribution on \lc and high
%diversity of inputs, we estabilish \reqs of input and output
%for each \lc, and find inputs that fulfil the \reqs. Given a
%\lc, a \req consists of search \req, transform
%\req and expansion \req. The search \req
%describes features and functionalities that we seek to have in
%inputs. \Model check each input if it satisfy the \req.

Table \ref{tab:specification} shows the search rules and the
transformation templates of all 11 linguistic capabilities we
implemented in \tool{}. \sw{will revise the rest of 3.1 after talking
  to Jaeseong... I think we can show all the specifications.}

\begin{table*}[t]
\small
\centering
\begin{tabular}{l|l}
  {\bf Linguistic capability}	 & {\bf Search rule and transformation template}\\
  \hline
  LC1: Short sentences with neutral & search: \{length: <10; include: neutral adjs \& neutral nouns; \\
  adjectives and nouns & exclude: pos adjs \& neg adjs \& pos nouns \& neg nouns; label: neutral\}  \\
  & transform: - \\
  \hline
  LC2: Short sentences with  & search: \{length: <10; include: pos adjs; exclude: neg adjs \& neg verbs \& neg nouns; label: pos\} \\
  sentiment-laden adjectives  & | \{length: <10; include: neg adjs; exclude: pos adjs \& pos verbs \& pos nouns \& neg verbs \& neg nouns; label: neg\}  \\
  & transform: - \\
  \hline
  LC3: Sentiment change over time, & search: \{label: pos\} | \{label: neg\} \\
  present should prevail & transform: \\
  \hline
  LC4: Negated negative should be & search: \{start: [This, That, These, Those] X [is, are]; label: neg\} \\
  positive or neutral & transform:  \\
  \hline
  LC5: Negated neutral should & search: \{start: [This, That, These, Those] X [is, are]; label: neutral\} \\
  still be neutral & transform: \\
  \hline
  LC6: Negation of negative at the & search: \{label: neg\} \\
  end, should be positive or neutral & transform: \\
  \hline
  LC7: Negated positive with neutral & search: \{length: <20; label: pos\} | \{length: <20; label: neutral\} \sw{Not sure about this rule.} \\
  content in the middle & transform: \\
  \hline
  LC8: Author sentiment is more & search: \{label: pos\} | \{label: neg\} \\
  important than of others & transform: \\
  \hline
  LC9: parsing sentiment in & search: \{label: pos\} | \{label: neg\} \\
  (question, yes) form & transform: \\
  \hline
  LC10: Parsing positive sentiment & search: \{label: pos\} \\
  in (question, no) form & transform: \\
  \hline
  LC11: Parsing negative sentiment & search: \{label: neg\} \\
  in (question, no) form & transform: \\
  \hline
\end{tabular}
\caption{Search rules and transformation templates for linguistic capabilities. \sw{Add transformation templates. May need to find a better specification language.}}
\label{tab:specification}
\end{table*}


\begin{figure}[t]
  \centering
  \lstinputlisting[language=json-pretty]{code/requirement_sa1.json}
  \vspace{-10pt}
  \caption{\SearchRequirementExampleFigCaption}
  \vspace{-10pt}
\end{figure}

\begin{figure}[t]
  \centering
  \subfloat[][\TransformRequirementExampleSubFigCaption]{\lstinputlisting[language=json-pretty]{code/requirement_sa2.json}}
  \\
  \subfloat[][\TransformTemplateExampleSubFigCaption]{\lstinputlisting[language=python]{code/requirement_sa2.py}}
  \\
  \caption{\TransformRequirementExampleFigCaption}
  \vspace{-10pt}
\end{figure}

Figure~\ref{fig:SearchReqEx} shows \lc of \SareqExOne. To evaluate
this \lc, the input is required to be short and have only \neu \adjs,
\neu \nns. In addition, the label needs to be \neu. Therefore, all
short natural \sents with only \neu \adjs and \neu \nns are available
to evaluate NLP models. In this work, the sentiment of the words for
the search are classified based on the sentiment scores from
\Swn~\cite{baccianella2010sentiwordnet}, a publicly available English
sentiment lexicons.  It provides lexical sentiment scores and the
sentiment word labels are categorized by implementing the rules
in~\cite{mihaela2017sentiwordnetlabel}. Next, transform \req explains
how the input and output needs to be tranfromed. Some \lc only accepts
heavily limited input distribution, and it is unlikely to be included
in searching dataset because of its high structural diversity, thus,
finding such \sents is costly. Therefore, our approach is to find
inputs by relaxing search requirement and transform the input to match
the target requirement of the \lc. In this work, the inputs are
transformed by word addition or perturbing the found inputs with \lc
dependent templates. The figure~\ref{fig:TransformReqEx} shows the
example of use of the template requirement. The \lc of \SareqExTwo in
the figure~\ref{fig:TransformReqSubEx} requires inputs to be the
negated \pstv \sents and the neutral expression in the middle. Rather
than searching \sents that match the input distribution of the \lc,
the \Model search \pstv and \neu inputs and combine them into negated
\pstv \sents. Figure~\ref{fig:TransformTempSubEx} illustrates template
for the \lc. According to the \lc, The value of ``sent1'' and
``sent2'' become each searched \neu and \pstv inputs respectively, and
the template completion generates new inputs that matches the target
\lc. In addition, the transformation of inputs also produce high
diversity in the inputs because of that from initially found
inputs. In this paper, we will denote the searched inputs in this
phase as seed inputs.

\subsection{Syntax-based Sentence Expansion}

The simple search rules and transformation templates used to generate
the seed sentences may limit the syntactic structures these seeds may
cover. To address this limitation, the syntax-based sentence expansion
phase extends the seed sentences to cover syntactic structures
commonly used in real-life sentences. Our idea is to differentiate the
parse trees between the seed sentences and the reference sentences
from a large real-world dataset. The extra PoS tags in the reference
parse trees are identified as potential syntactic elements for
expansion and inserted into the seed sentences as masks. We then use
masked language model to suggest the fill-ins. If the resulting
sentences still conform to the linguistic capability specification,
they are added to \tool{}'s test suite. \sw{May have some redundancy
  and inconsistency with the overview part.}

\subsubsection{Syntax Expansion Identification}

Algorithm \ref{alg:diff} shows how masks are identified for each seed
sentence.  It takes the parse trees of the seeds, generated by the
Berkeley Neural
Parser~\cite{kitaev2018seedparser,kitaev2019seedparser}, and a
reference context-free grammar (CFG) from the Penn Treebank corpus
dataset \cite{} as inputs.  The reference CFG is learned from a large
dataset \cite{} that is representative of the distribution of
real-world language usage.  The algorithm identifies the discrepancy
between the seed syntax and the reference grammar to decide how a seed
can be expanded.
%This phase builds \pcfg from the reference corpus. \cfg is
%constructed by parsing sentences in the corpus and extracting
%\prodrs. In addition, the probability of \cfg is estimated by its
%frequency over corpus. The output of this phase is the constructed
%\pcfg, and it is compared with seed parse trees. For our
%implementation, we build the \pcfg from the Penn Treebank corpus
%dataset.


%\paragraph{Seed parsing.}  To expand seed sentence and generate
%fluent and faithful sentence used for evaluation, \Model studies
%structure of each seed input for its expansion.  Seed parsing takes
%each seed as input, and outputs its parse tree of the seed, using the
%Berkeley Neural
%Parser~\cite{kitaev2018seedparser,kitaev2019seedparser}.

%\paragraph{Reference parsing.}  We take a large-scale real-world
%dataset as reference for sentence expansion.

\begin{algorithm}
  \caption{production differentiation}
  \begin{algorithmic}[1]
    \State \textbf{Input:} Parse trees of seed sentences $\textbf{S}$,
    reference \pcfg$\textbf{R}$
    \State \textbf{Output:} Set of expanded production rule $\textbf{P}$
    \For {seed from $S$}
      \For {each production rule $seed\_prod$ from seed}
        \State $seed\_lhs = seed\_prod.lhs$
        \State $seed\_rhs = seed\_prod.rhs$
        \For {$ref\_rhs$, $ref\_prob$ from $R[seed\_lhs]$}
          \If {$seed\_rhs$ is superset of $ref\_rhs$}
            \State $parent = seed\_prod.parent$
            \State $prob = ref\_prob$
            \While {$parent$ is not empty}
              \For {$par\_rhs$, $par\_prob$ from $R[parent.lhs]$}
                \If {$parent.rhs == par\_rhs$}
                  \State $prob = prob \cdot par\_prob$
                  \State $parent = parent.parent$
                  \State break
                \EndIf
              \EndFor  
            \EndWhile
            \State $P.add([ref\_prod, prob])$
          \EndIf
        \EndFor
      \EndFor
    \EndFor
  \end{algorithmic} 
\end{algorithm}


For each production of in each seed's parse tree (lines 3 and 4), we
extract its non-terminal at the left-hand-side (line 5), $s\_lhs$, and
the grammar symbols at the right-hand-side (line 6), $s\_rhs$. In line
7, the algorithm iterates through all productions in the reference
context-free grammar and match these that have the same non-terminal
at the left-hand-side as $s\_lhs$.  The right-hand-side of each
matched production is called $r\_rhs$.  If $s\_rhs$ consists of a
subset of the grammar symbols in $r\_rhs$ (line 8), the additional
symbols in the $r\_rhs$ are inserted as masks in the parse tree of
seed sentence, in their respective positions in the expanded
production.  The left to right traversal of the leaves of an expanded
parse tree forms a masked sentence.  Lastly, we randomly select $k$
masked sentences for the next sentence expansion and validation phase.
\sw{Add justification: why we need to select k masked sentences
  (performance?) and why random makes sense.}

\begin{figure}
  \centering
  \includegraphics[width=\linewidth]{figs/expansion.pdf}
  \caption{Example of masked sentence generation. \sw{Expansion: Or both {MASK}. -> Masked sentence: Or both {MASK.}}}
%  Expansion of the seed sentence ``Or
%both.''. For a \prodr in seed ``NP->[DT]'' on the left, the \prodr of
%``NP->[DT, NNS]'' is found in reference. Thus, the component NNS can be expanded in the
%seed and ``NP->[DT]'' is replaced with ``NP->[DT, NNS]'' and it generates new
%expanded sentence on the right.
\label{fig:ExpEx}
\end{figure}

Figure \ref{fig:ExpEx} shows an example using Algorithm \ref{alg:diff}
to generate a masked sentence. The sentence ``Or both." is a seed of
\sw{which?} linguistic capability.  \sw{Not sure about the quality of
  the example. Maybe use a better seed for this example?}  The tree on
the left shows the parse tree of this seed; it consists of two
productions: ``FRAG->[CC, NP, .]" and ``NP->[DT]".  When matching the
left-hand-side non-terminal of the second production (i.e., ``NP") in
the reference CFG, we found that it includes a production ``NP->[DT,
  NNS]" which has an additional symbol ``NNS" on the right-hand-side.
The algorithm thus expands the parse tree with this symbol, shown on
the right.  The masked sentence ``Or both \{MASK\}." is the result of
the left-to-right traversal of this expanded parse tree.

%Given the seed parse tree and reference \pcfg, production
%differentiation phase suggests structural expansion candidates on the
%seed input. this phase aims to analyze which structural components
%and where they can be added into the seed structure for its
%expansion. To do so, we explore reference \prodrs comparing it with
%each \prodr used in seed input. This results in the phase described
%in Algorithm~\ref{code:ProdDiffAlg}. For each production rule in seed
%inputs ($seed\_prod$), it searches production rules in reference
%($ref\_prod$) which it has same non-terminal on the \lhs
%($seed\_prod.lhs==ref\_prod.lhs$) and superset of \rhs of the seed
%production rule ($seed\_prod.rhs \subset ref\_prod.rhs$).  As we
%assume that the reference \cfg is built from real world data
%distribution, the elements in the complement set ($ref\_prod.rhs -
%seed\_prod.rhs$) become an expansion candidate which can be expanded
%from the $seed\_prod.rhs$ found in real world. In addition, the
%measure of how consistent the production rule is with the given seed
%structure is given in its probability of the reference production
%rule ($ref\_prob$) multiplied by that of parents of $seed\_prod$. The
%expansion candidate consists of terminal or nonterminal symbols. When
%there is a phrase-level or clause-level nonterminal symbol, \eg noun
%phrase, it needs to be expanded and replaced with word-level
%nonterminal or terminal symbols to generate the expansion
%candidate. The number of feasible replacement is unbounded because of
%its high degree of freedom. Therefore, in this work, we focus on the
%expansion candidate with only the word-level nonterminal or terminal
%symbols for the effectiveness of \Model. Lastly, the expanded
%component is replaced with the mask token for the next phase. The
%example of the expansion is illustrated in
%Figure~\ref{fig:ExpEx}. The ``NP->[DT]'' is queried into reference,
%and ``NP->[DT,NNS]'' is identified as its expansion candidate since
%the \rhs of ``NP->[DT,NNS]'' is superset of that of ``NP->[DT]''. the
%component of NNS is replaced with mask token in
%sentence-level. Therefore, the ``Or both \{MASK\}.'' is suggested for
%the next phase.

\subsubsection{Sentence Expansion and Validation}

In this phase, the words to fill in the masks in the masked sentences
are suggested by the BERT pretrained model \cite{}.  The BERT model
suggests word for the mask symbol according to its context \sw{what
  does context mean here?} around in sentence.  \sw{Algorithm 1 does
  not require we only have one mask in the masked sentence. Can the
  model suggest multiple words at the same time?}  \sw{Say a bit more
  about the BERT model suggestion. E.g., it may suggest multiple words
  for the same mask but they are ranked?}

Because BERT model is not aware of the linguistic capability
specification and the grammar symbol in the expanded parse tree, an
expanded sentence using the suggested words may no longer satisfy the
linguistic capability specification. Therefore, we perform validation
on the suggested words and only accept them if the following three
criteria are met.

First, the PoS tag of the suggested word must match the PoS tag of the
expanded symbol in the parse tree. For the example in Figure
\ref{fig:ExpEx}, the masked symbol is a ``NNS" (i.e., plural noun);
thus, the suggested word must also be a ``NNS". \sw{Say how we obtain
  the PoS tag of a suggested word.}  \sw{Also, expand Figure 4 to
  actually include the example of the suggested words?  That way, we
  can easily discuss each validation criteria along with the example.}
Second, we require that the sentiment of the expanded sentence is the
same as the seed sentence. To ensure this, the suggested words must be
neutral.  \sw{Should we present this as part of specification, called
  expansion rule?}  Third, the expanded sentence must still conform to
the specification of the seed's linguistic capability specification.
\sw{It is not clear to me how we check with search rules and
  transformation templates.}

%The suggested word are validated by three criteria. First, The tag of
%POS must be matched with that suggested from the production
%differentiation phase. In the example in the Figure~\ref{fig:ExpEx},
%the mask token comes from the structural component of NNS, plural
%noun. Therefore, the BERT suggested words must be also tagged with
%the NNS. Accordingly, every words of the NNS are only
%available. Second, \Model focuses on the \sa task, and it assume that
%the suggested words must not change sentiment of its input and must
%preserve its consistency of original sentiment label. Therefore, we
%only accept the \neu words for the expansion. Third, the expanded
%sentences with the BERT suggested words must be appropriate for
%evaluating NLP models on the target \lc, and the sentences must pass
%the \req of the \lc. In this work, the BERT suggestions are validated
%by the three criteria.

\subsubsection{Sentence Selection}

\sw{Use BERT score as heuristic to select expanded sentences. Motivate
  why.}
