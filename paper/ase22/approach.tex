\section{Specification- and Syntax-based Linguistic Capability Testing}
\label{sec:approach}

\begin{figure*}
  \centering
  \includegraphics[width=\linewidth]{figs/overview.pdf}
  \caption{\OverviewFigCaption}
\end{figure*}

\sw{Some paragraphs in this overview part should go earlier in the
  paper. I am just writing them here for now as I don't think we have
  them in the intro/background yet.}  We design and implement
\emph{Specification- and Syntax-based Linguistic Capability Testing
  (\tool{})} to automatically generate test cases to test the
robustness of sentiment analysis models. We identify four goals for a
large and effective test suite:

\begin{description}
\item[{\bf G1}] the test suite should contain realistic \sents;
\item[{\bf G2}] the test suite should cover diverse syntactic structures;
  \item[{\bf G3}] each test case should be
categorized into a \lc;
\item[{\bf G4}] the label of each test case should be
automatically and accurately defined.
\end{description}

\Cklst's templates generate complete and realistic \sents, and each
template maps to a \lc, satisfying {\bf G1} and {\bf
  G3}. But \Cklst only uses \sw{X} manually created templates to
generate its test suite; all test cases generated by the same template
share the same syntactic structure, thus violating {\bf G2}. In
addition, the label of each \Cklst test case has to be decided
manually, associated with each template, violating {\bf G4}.

We present \tool{}, a new \lc test case generation
tool, that satisfies all of these criteria.  \sw{I could not summarize
  a cohesive idea that drives our design. Leaving it here to fill in.
  Also, I felt my writing below still lacks justification for some
  design choices (e.g., why we do the differentiation).}
Figure~\ref{fig:overview} shows the overview of \tool{}, which
consists of two phases.  The \emph{specification-based seed
  generation} phase performs rule-based searches from a real-world
dataset ({\bf G1}) and template-based transformation to obtain the
initial seed \sents.  The search rules (e.g., search for \neu
\sents that do not include any \pstv or \ngtv words) and
transformation templates (e.g., \sw{add an example} \jl{\sent
  negation}) are defined in the \emph{\lc
  specifications}, which guarantee that each resulting seed conforms
to a specific \lc ({\bf G3}) and is labelled
correctly ({\bf G4}).

The \emph{syntax-based \sent expansion} phase expands the seed
\sents with additional syntactic elements (i.e., words
\sw{more?}\jl{and production ruls in \cfg}) to cover many real-world
syntactic structures ({\bf G2}). It first performs a syntax analysis
to identify the \pos (PoS) tags that can be inserted to each
seed, by comparing the PoS parse trees between the seed \sent and
many other \sents from a large reference dataset. Each identified
tag is inserted into the seed as a \emph{mask}. It then uses an NLP
recommendation model (i.e., BERT \cite{}) to suggest possible
words. If a resulting \sent is validated to be consistent with the
specification which additionally defines the rules for expansion
(e.g., the expanded word should be \neu), {\bf G3} and {\bf G4} are
still satisfied.  Last, because some validated \sents may include
unacceptable suggested words given the context \sw{is this the
  right motivation to do selection?} \jl{I edited the \sent. please
  let me know if it is clear}, we use a heuristic (i.e., the
confidence score from the NLP recommendation model) to select the more
realistic context-aware expanded \sents into \tool{}'s test suite.

We now describe each phase of \tool{} in detail.

%\Model generates input \sents with the following phases illustrated
%in \ref{fig:OverallModel}: 1. search phase searches seed \sents
%according to its \req of \lc, 2. seed parsing phase parses the found
%seed \sents and extract their \cfg, 3. reference parsing phase
%collects \pcfg from large corpus, 4. production differentiation phase
%identifies structural expansion candidates for input expansion, and
%5.\sent selection phase generates natural expanded \sent. In this
%section, we provide more details on each phase.

\subsection{Specification-based Seed Generation}
The seed generation phase of \tool starts by searching \sents in a
real-world dataset that match the rules defined in the \lc
specification, and then transforming the matched \sents using
templates to generate seed \sents that conform to individual
\lcs. The reasons for this design choice are twofold.  First, while
generally judging which \lc any \sent falls into and which label it
should have is infeasible, there exist simple rules and templates to
allow classifying the resulting \sents into individual \lcs and
with the correct labels, with high confidence.  This enables us to
test each \lc individually.  Second, searching from a real-world
dataset ensures that the \sents used as test cases for testing \lcs
are realistic and diverse. The diverse test cases are more likely to
achieve a high coverage of the target model's functionality in each
\lc, thus detecting more errors.
%
In this phase, \tool first search and selects \sents applicable to
the \lc in a given real-world dataset with search rules. In case that
the search rules only fulfill portion of the \lc specifications, the
selected \sents are not yet appropriate to become seed, we
transform the selected \sents into seed \sents using the
heuristic templates.
%
Table~\ref{tab:specification} shows the search rules and the
transformation templates of all 11 linguistic capabilities we
implemented in \tool. The first column shows the \lc type and its
description, and the second column shows the search rule and
transformation template used in each \lc. For LC1 and LC2, the NLP
models are evaluated in the scope of short \sents with selective
sentiment words. It does not require any guidance for its
transformation because the search rule alone is sufficient to conform
to the \lcs. On the other hand, search rules of LC3 and LC4 are not is
not enough to match their \lc specification, thus \tool uses heuristic
templates to conform the found \sents to the \lc. For example, the
selected \sents becomes seeds by preturbing them with the templates
and by negating the selected demonstrative \sents to conform to LC3
and LC4 respectively (see the third and forth rows in
Table~\ref{tab:specification}).
\sw{@Jaeseong: revise the rest of 3.1 based on
  Table~\ref{tab:specification}. I commented out the old text but it
  is still in the tex file.} \jl{I revised and added the explanation}

%The search phase in \Model searches inputs in dataset and selects
%subset of input \sents in the dataset that meets the \lc
%\req. The idea behind this phase is that input distribution of
%\lc is important to generate inputs relevant to \lc. \Lc explains
%expected behaviors of NLP model on specific types of input and
%output. The NLP model is evaluated on how much it performs on the
%input and output. Thus, \lc introduces the constraints of the input
%data. Input data from the constrained distribution are only qualified
%to be used for evaluating the NLP model on the \lc.  In addition,
%diversity in inputs is important to evaluate NLP models on the
%\lc. Inputs that differ are more likely to cover the NLP model
%behavior, and more coverage increases trustworthiness of the
%evaluation. To generate inputs from same distribution on \lc and high
%diversity of inputs, we estabilish \reqs of input and output
%for each \lc, and find inputs that fulfil the \reqs. Given a
%\lc, a \req consists of search \req, transform
%\req and expansion \req. The search \req
%describes features and functionalities that we seek to have in
%inputs. \Model check each input if it satisfy the \req.

\InputWithSpace{tables/lc-requirement-table}

%\begin{figure}[t]
%  \centering
%  \lstinputlisting[language=json-pretty]{code/requirement_sa1.json}
%  \vspace{-10pt}
%  \caption{\SearchRequirementExampleFigCaption}
%  \vspace{-10pt}
%\end{figure}
%
%\begin{figure}[t]
%  \centering
%  \subfloat[][\TransformRequirementExampleSubFigCaption]{\lstinputlisting[language=json-pretty]{code/requirement_sa2.json}}
%  \\
%  \subfloat[][\TransformTemplateExampleSubFigCaption]{\lstinputlisting[language=python]{code/requirement_sa2.py}}
%  \\
%  \caption{\TransformRequirementExampleFigCaption}
%  \vspace{-10pt}
%\end{figure}
%
%Figure~\ref{fig:SearchReqEx} shows \lc of \SareqExOne. To evaluate
%this \lc, the input is required to be short and have only \neu \adjs,
%\neu \nns. In addition, the label needs to be \neu. Therefore, all
%short natural \sents with only \neu \adjs and \neu \nns are available
%to evaluate NLP models. In this work, the sentiment of the words for
%the search are classified based on the sentiment scores from
%\Swn~\cite{baccianella2010sentiwordnet}, a publicly available English
%sentiment lexicons.  It provides lexical sentiment scores and the
%sentiment word labels are categorized by implementing the rules
%in~\cite{mihaela2017sentiwordnetlabel}. Next, transform \req explains
%how the input and output needs to be tranfromed. Some \lc only accepts
%heavily limited input distribution, and it is unlikely to be included
%in searching dataset because of its high structural diversity, thus,
%finding such \sents is costly. Therefore, our approach is to find
%inputs by relaxing search requirement and transform the input to match
%the target requirement of the \lc. In this work, the inputs are
%transformed by word addition or perturbing the found inputs with \lc
%dependent templates. The figure~\ref{fig:TransformReqEx} shows the
%example of use of the template requirement. The \lc of \SareqExTwo in
%the figure~\ref{fig:TransformReqSubEx} requires inputs to be the
%negated \pstv \sents and the \neu expression in the middle. Rather
%than searching \sents that match the input distribution of the \lc,
%the \Model search \pstv and \neu inputs and combine them into negated
%\pstv \sents. Figure~\ref{fig:TransformTempSubEx} illustrates template
%for the \lc. According to the \lc, The value of ``sent1'' and
%``sent2'' become each searched \neu and \pstv inputs respectively, and
%the template completion generates new inputs that matches the target
%\lc. In addition, the transformation of inputs also produce high
%diversity in the inputs because of that from initially found
%inputs. In this paper, we will denote the searched inputs in this
%phase as seed inputs.

\subsection{Syntax-based \Sent Expansion}

The simple search rules and transformation templates used to generate
the seed \sents may limit the syntactic structures these seeds may
cover. To address this limitation, the syntax-based \sent expansion
phase extends the seed \sents to cover syntactic structures
commonly used in real-life \sents. Our idea is to differentiate the
parse trees between the seed \sents and the reference \sents
from a large real-world dataset. The extra PoS tags in the reference
parse trees are identified as potential syntactic elements for
expansion and inserted into the seed \sents as masks. We then use
masked language model to suggest the fill-ins. If the resulting
\sents still conform to the \lc specification,
they are added to \tool{}'s test suite. \sw{May have some redundancy
  and inconsistency with the overview part.}

\subsubsection{Syntax Expansion Identification}

Algorithm \ref{alg:diff} shows how masks are identified for each seed
\sent.  It takes the parse trees of the seeds, generated by the
Berkeley Neural
Parser~\cite{kitaev2018seedparser,kitaev2019seedparser}, and a
reference context-free grammar (CFG) from the Penn Treebank corpus
dataset \cite{} as inputs.  The reference CFG is learned from a large
dataset \cite{} that is representative of the distribution of
real-world language usage.  The algorithm identifies the discrepancy
between the seed syntax and the reference grammar to decide how a seed
can be expanded.
%This phase builds \pcfg from the reference corpus. \cfg
%is constructed by parsing \sents in the corpus and extracting
%\prodrs. In addition, the probability of \cfg is estimated by
%its frequency over corpus. The output of this phase is the constructed
%\pcfg, and it is compared with seed parse trees. For our
%implementation, we build the \pcfg from the Penn Treebank corpus dataset.


%\paragraph{Seed parsing.}
%To expand seed \sent and generate fluent and faithful \sent
%used for evaluation, \Model studies structure of each seed input for
%its expansion.
%Seed parsing takes each seed as
%input, and outputs its parse tree of the seed, using the Berkeley Neural
%Parser~\cite{kitaev2018seedparser,kitaev2019seedparser}.

%\paragraph{Reference parsing.}
%We take a large-scale real-world dataset as reference for \sent expansion.

\begin{algorithm}
  \caption{production differentiation}
  \begin{algorithmic}[1]
    \State \textbf{Input:} Parse trees of seed sentences $\textbf{S}$,
    reference \pcfg$\textbf{R}$
    \State \textbf{Output:} Set of expanded production rule $\textbf{P}$
    \For {seed from $S$}
      \For {each production rule $seed\_prod$ from seed}
        \State $seed\_lhs = seed\_prod.lhs$
        \State $seed\_rhs = seed\_prod.rhs$
        \For {$ref\_rhs$, $ref\_prob$ from $R[seed\_lhs]$}
          \If {$seed\_rhs$ is superset of $ref\_rhs$}
            \State $parent = seed\_prod.parent$
            \State $prob = ref\_prob$
            \While {$parent$ is not empty}
              \For {$par\_rhs$, $par\_prob$ from $R[parent.lhs]$}
                \If {$parent.rhs == par\_rhs$}
                  \State $prob = prob \cdot par\_prob$
                  \State $parent = parent.parent$
                  \State break
                \EndIf
              \EndFor  
            \EndWhile
            \State $P.add([ref\_prod, prob])$
          \EndIf
        \EndFor
      \EndFor
    \EndFor
  \end{algorithmic} 
\end{algorithm}


For each production of in each seed's parse tree (lines 3 and 4), we
extract its non-terminal at the left-hand-side (line 5), $s\_lhs$, and
the grammar symbols at the right-hand-side (line 6), $s\_rhs$. In line
7, the algorithm iterates through all productions in the reference
context-free grammar and match these that have the same non-terminal
at the left-hand-side as $s\_lhs$.  The right-hand-side of each
matched production is called $r\_rhs$.  If $s\_rhs$ consists of a
subset of the grammar symbols in $r\_rhs$ (line 8), the additional
symbols in the $r\_rhs$ are inserted as masks in the parse tree of
seed \sent, in their respective positions in the expanded production.
The left to right traversal of the leaves of an expanded parse tree
forms a masked \sent.  Lastly, due to the inefficient cost of
accessing full list of the masked \sents, we randomly select $k$
masked \sents for the next \sent expansion and validation phase when
the masked \sents are more than maximum number of masked \sents. The
random sampling is unbiased approach since it gives same chance to be
chosen. Thus, the random sample becomes representative of the
population of the masked \sents, and it efficiently shows the
usefulness of the \tool.  \sw{Add justification: why we need to select
  k masked \sents (performance?)  and why random makes sense.} \jl{I
  added the statement for it}

\begin{figure*}[t]
  \centering
  \includegraphics[scale=0.7]{figs/running_example.pdf}
  \caption{\RunningExCaption}
%  Expansion of the seed \sent ``Or
%both.''. For a \prodr in seed ``NP->[DT]'' on the left, the \prodr of
%``NP->[DT, NNS]'' is found in reference. Thus, the component NNS can be expanded in the
%seed and ``NP->[DT]'' is replaced with ``NP->[DT, NNS]'' and it generates new
%expanded \sent on the right.
\end{figure*}

\paragraph{Running example.} Figure~\ref{fig:ExpEx} shows an example using Algorithm \ref{alg:diff}
to generate a masked \sent. The \sent ``Or both." is a seed of \lc of
\SareqExOne.  The tree on the left shows the parse tree of this seed;
it consists of two productions: ``FRAG->[CC, NP, .]" and ``NP->[DT]".
When matching the left-hand-side non-terminal of the second production
(i.e., ``NP") in the reference CFG, we found that it includes a
production ``NP->[DT, NNS]" which has an additional symbol ``NNS" on
the right-hand-side.  The algorithm thus expands the parse tree with
this symbol, shown in the second tree.  The masked \sent ``Or both
\{MASK\}." is the result of the left-to-right traversal of this
expanded parse tree.

%Given the seed parse tree and reference \pcfg, production
%differentiation phase suggests structural expansion candidates on the
%seed input. this phase aims to analyze which structural components and
%where they can be added into the seed structure for its expansion. To
%do so, we explore reference \prodrs comparing it with each \prodr used
%in seed input. This results in the phase described in
%Algorithm~\ref{code:ProdDiffAlg}. For each production rule in seed
%inputs ($seed\_prod$), it searches production rules in reference
%($ref\_prod$) which it has same non-terminal on the \lhs
%($seed\_prod.lhs==ref\_prod.lhs$) and superset of \rhs of the seed
%production rule ($seed\_prod.rhs \subset ref\_prod.rhs$).  As we
%assume that the reference \cfg is built from real world data
%distribution, the elements in the complement set ($ref\_prod.rhs -
%seed\_prod.rhs$) become an expansion candidate which can be expanded
%from the $seed\_prod.rhs$ found in real world. In addition, the
%measure of how consistent the production rule is with the given seed
%structure is given in its probability of the reference production rule
%($ref\_prob$) multiplied by that of parents of $seed\_prod$. The
%expansion candidate consists of terminal or nonterminal symbols. When
%there is a phrase-level or clause-level nonterminal symbol, \eg noun
%phrase, it needs to be expanded and replaced with word-level
%nonterminal or terminal symbols to generate the expansion
%candidate. The number of feasible replacement is unbounded because of
%its high degree of freedom. Therefore, in this work, we focus on the
%expansion candidate with only the word-level nonterminal or terminal
%symbols for the effectiveness of \Model. Lastly, the expanded
%component is replaced with the mask token for the next phase. The
%example of the expansion is illustrated in
%Figure~\ref{fig:ExpEx}. The ``NP->[DT]'' is queried into
%reference, and ``NP->[DT,NNS]'' is identified as its expansion
%candidate since the \rhs of ``NP->[DT,NNS]'' is superset of that of
%``NP->[DT]''. the component of NNS is replaced with mask token in
%\sent-level. Therefore, the ``Or both \{MASK\}.'' is suggested for
%the next phase.

\subsubsection{\Sent Expansion and Validation}

In this phase, the words to fill in the masks in the masked \sents are
suggested by the BERT pretrained model~\cite{devlin2019bert}.  The
BERT is a transformer-based \nl model. It is pretrained on two tasks
of masked token prediction and next sentence prediction. As a result
of the training process, the BERT model suggests word for the mask
token according to its surrounding context in \sent. For each masked
token, multiple words are suggested ranked by their confidence scores.
\sw{Algorithm 1 does not require we only have one mask in the masked
  \sent. Can the model suggest multiple words at the same time?}
\jl{yes. it can predict multiple masked tokens at the same time.}
\sw{Say a bit more about the BERT model suggestion. E.g., it may
  suggest multiple words for the same mask but they are ranked?}
\jl{I added the explanation of BERT}
%
Because BERT model is not aware of the \lc
specification and the grammar symbol in the expanded parse tree, an
expanded \sent using the suggested words may no longer satisfy the
\lc specification. Therefore, we perform validation
on the suggested words and only accept them if the following three
criteria are met.

First, the PoS tag of the suggested word must match the PoS tag of the
expanded symbol in the parse tree. For the example in
Figure~\ref{fig:ExpEx}, the masked symbol is a ``NNS" (i.e., plural
noun); thus, the suggested word must also be a ``NNS". In this work,
we use \spacy, a free open-source library for \nlp, for extracting PoS
tags for each suggested word.\sw{Say how we obtain the PoS tag of a
  suggested word.}\jl{I added it} Second, it is required that the
sentiment of the expanded \sent becomes the same as the seed \sent. To
ensure this, the suggested words must be \neu.  \sw{Should we present
  this as part of specification, called expansion rule?} \jl{I did it
  because I think that there is no issue on mentioning it and that is
  how we assumed and did} Third, we additionally verify that the
expanded \sents satisfied the same search rules for the seed
\sent. Our goal for generating the expanded \sents is to use them for
evaluating the \sa models on the associated \lc in addition to the
seed \sent. It is only achieved when the expanded \sents are also met
with the same search rules for the \lc. For LC1 as an example, the
expanded \sent must still conform to the specification of the seed's
\lc specification. Therefore, the expanded \sents are required to be
short and to only have \neu adjectives and \nns. \sw{Say why we use
  the third criteria only for LC1 and LC2.}  \jl{I added the
  explanation}

\paragraph{Running example.} The third step in Figure~\ref{fig:ExpEx} shows the words suggested
by BERT. For this masked \sent, BERT suggested six words. Each word
is associated with the confidence score provided by BERT, the PoS tag,
and the sentiment. Among the six words, only ``ways'' and ``things''
are validated by \tool{} because they have the Pos tag ``NNS'' and are
\neu. In addition, it is found that both \sents meets the search
rule of the associated \lc of \SareqExOne. In the end, two \sents
of ``Or both ways'' and ``Or both things'' are generated.  \sw{Do we
  also check if the expanded \sent meets the
  specification?}\jl{Yes. I added the explanation of validation of \lc
  \req}

%The suggested word are
%validated by three criteria. First, The tag of POS must be matched with
%that suggested from the production differentiation phase. In the
%example in the Figure~\ref{fig:ExpEx}, the mask token comes from the
%structural component of NNS, plural noun. Therefore, the BERT
%suggested words must be also tagged with the NNS. Accordingly, every
%words of the NNS are only available. Second, \Model focuses on the \sa
%task, and it assume that the suggested words must not change sentiment
%of its input and must preserve its consistency of original sentiment
%label. Therefore, we only accept the \neu words for the
%expansion. Third, the expanded \sents with the BERT suggested words
%must be appropriate for evaluating NLP models on the target \lc, and the
%\sents must pass the \req of the \lc. In this work, the BERT
%suggestions are validated by the three criteria.

\subsubsection{\Sent Selection}
\sw{@Jaeseong: Add how we select expanded \sents and motivate why.}
After 
\paragraph{Running example.} \sw{Refer to the example to say how we
  use the BERT score to select.} \jl{I dont think we need this
  subsection of \sent selection it is already explained at the
  previous stage with running example.}
