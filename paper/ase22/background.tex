\section{Background}
\label{sec:background}

%% \sw{We should motivate the testing of linguistic capabilities. Give
%%   background of the work that has been done in CheckList: what
%%   capabilities they support and their limitations. Maybe find a
%%   motivating example?}

\begin{figure}[t]
 \centering
 \lstinputlisting[language=python-pretty]{code/checklist_template.py}
 \vspace{-10pt}
 \caption{\CklstTemplateFigCaption}
 \vspace{-10pt}
\end{figure}

In this section, we provide a brief background on \Cklst testcase
generation via an example. Quality of software is verified by ensuring
the proper working of all functionalities without knowing the internal
workings of the software. Knowing performances of model on the
multiple functionalities provides users with better understanding and
debugging the software. The same principle applies to the NLP model.
traditional NLP model evaluation relying on a test set is lack of
specification of model functionality. However, In NLP domain, there
are many phenomena on linguistic input such as negation,
questionization. Given a NLP task, the phenomena determine
task-relevant output. Traditional evaluation method neglects them,
thus, it becomes less efficient to detect and analyze which aspect the
model yields unexpected outcome. To tackle this limitation, \Cklst
introduces task-dependent \lcs for monitoring model performance on
each \lc. It assumes that the linguistic phenomena can be represented
into the model behaviors as they provide what input and output are
desired and how the model works with them. For each \lc, \Cklst makes
testcase templates and generate sentences by filling-in the value for
each placeholder. We show an example of the templates in
Figure~\ref{code:TempEx}. The templates in the figure is used for
evaluating a \sa model on \SareqExThree. For the templates defined
from line~\ref{code:tempex:template:1} to \ref{code:tempex:template:2}
have placeholders such as ${it}$, ${air_noun}$, ${pos_adj}$. Values
for the placeholders are defined at
line~\ref{code:tempex:itbe},~\ref{code:tempex:airnoun:1}
and~\ref{code:tempex:posadj:1}. Ater all, all combinations of the
values of placeholders in a template are filled-in the template, and
the senteneces are generated such as ``The flight is good'', ``That
airline was happy'' and so on. Finally, these are used for evaluation
of \lc. Despite its simplicity of testcase generation, it still has
limitations: it first relies on manual work for defining template
structure and its values. Therefore, the manual work for testcase
generation keeps the process costly. Extended from it, Second, such
manual work produces the imitative forms between test cases, and it is
likely to introduce bias on the testcases. We will show that these
observations contribute to the performance of our models.
