\section{Experimental Setup}
\label{sec:experiment}
%
In this section, we present the setup of our experiments to evaluate
the effectiveness of \tool. We answer the following research
questions (RQs):

\begin{enumerate}[label=\textbf{RQ\arabic*}]
% \item \label{rq:three}: \textbf{Test Case Diversity.} Can \tool generate more diverse test cases
%   than \Cklst?
\item \label{rq:three}: \textbf{Test Efficiency.} Can \tool generated test cases evaluate an NLP model more efficiently?
\item \label{rq:one}: \textbf{Test Oracle.}  Can \tool generate test sentences with correct sentiment labels?
\item \label{rq:two}: \textbf{Capability Testing.} Are \tool generated test sentences correctly categorized into a \lc?
% \item \label{rq:four}: \textbf{Usefulness of Test Inputs.} Can \tool be useful to find root causes of bugs
%   in the \sa models?
%% \item \label{rq:three}: How effective is our new test case generation
%%   using \cfg expansion? % ablation study
\end{enumerate}

%To answer the RQs, we need to show: (\romnum{1}) correctness of testcase to
%evaluate its target \lc.  (\romnum{2}) effect of testcase distribution on
%finding bugs; (\romnum{3}) degree of execution of a \sa model on testcases;
%and (\romnum{4}) ability to guide to find cause of bug in a \sa
%model.

\subsection{Experimental Subjects}
\paragraph*{\textbf{NLP Models \& Dataset}}
We evaluate our approach on three \sa models and two \hsd models. All
are learning-based models released from the Hugging Face centralized
model hub:\footnote{https://huggingface.co/models} \texttt{\Bert}
(\bertsamodel), \texttt{\Roberta} (\robertasamodel), and
\texttt{\Dbert} (\disbertsamodel~\cite{sanh2019distilbertad}) for \sa
task, and \texttt{\hsdBert} (\berthsdmodel~\cite{aluru2020deep}) and
\texttt{\hsdRoBerta} (\robertahsdmodel) for \hsd.

These models were pre-trained on English language using a masked
language modeling (MLM) objective, and were fine-tuned on the \sa and
\hsd task. In this experiment, we used the \Sst~\cite{socher2013sst}
and \Hatexp~\cite{mathew2021hatexplain} datasets, for \sa and \hsd
respectively, for searching the seeds in \tool. \Sst is a corpus of
movie review, consisting of 11,855 sentences with sentiment scores. We
split the scores into ranges [0, 0.4], (0.4, 0.6] and (0.6, 1.0] to
    assign them negative, neutral and positive labels,
    respectively. \Hatexp is a hate speech dataset collected from
    Twitter and Gab. It consists of 20,148 sentences (9,055 from
    Twitter and 11,093 from Gab). Each data has sentence, class of
    hate speech, target community, and rationale for the labelling
    decision.  In addition, \Swn~\cite{baccianella2010sentiwordnet} is
    a publicly available dataset for English sentiment lexicons.  We
    used \Swn as both the generation and the expansion domain
    knowledge as shown in Figure \ref{fig:overview}.

\paragraph*{\textbf{Comparison Baseline}}
We compared \tool with
\Cklst,\footnote{https://github.com/marcotcr/checklist} a manual
template and \lc based approach to generate test cases. In this
evalaution, we used \Cklst's \sa test cases that are generated from
its publicly available Jupyter Notebook implementation. \jl{Do we need
  to explain why we used checklist only as a baseline?}

\subsection{Experimental Process}

\paragraph*{\textbf{RQ1}}
Recall that \Cklst relies on significant manual efforts and may not
generate comprehensive test cases in a linguistic capability. \tool,
instead, automatically generates test cases based on a search dataset
and the syntax in a large reference corpus.  We expect \tool can
generate more efficient test suite than \Cklst for finding errors with
regards to quantity and diversity of test suite.

We first evaluate the three \sa models and two \hsd models by testing
them on the \tool test cases, reporting number of test cases and each
model's failure rate in each linguistic capability. Specifically, for
each \lc, we expand test cases from all generated seeds.
% We selected subsets of seeds to be used for expansion because the
% syntax-based expansion phase may be expensive to run all the
% generated seeds.
We test the five models using both seed and expanded test cases. 
% To account for the randomness in random seed selection, we repeated
% these experiments 3 times and report median number of test cases and
% each model's failure rate.

We use two metrics to compare the diversity between the \tool
generated seeds and the \Cklst test cases.

\paragraph*{\selfbleu} We reuse the input diversity metric, 
called \selfbleu, that Zhu \etal
introduced~\cite{zhu2018texygen}. \bleu evaluates the token-level
similarity.
% Regarding that \selfbleu takes each sentence as hypothesis and rest
% in a collection of textual data as reference, we calculate \bleu
% scores for every pairs of hypothesis and each reference sentence.
The \selfbleu is defined as the average \bleu scores over all
reference sentences.
% Since the \bleu score ranges from 0 as the least similar inputs to 1
% as the most similar inputs,
A higher \selfbleu score indicates less diversity in the test
suite. In the experiment, we collected 50, 100 and 200 randomly
selected \tool seeds and reported the median \selfbleu score over
three trials for each group of seeds.
% The median \selfbleu over 3 trials of the random collection of the
% 50, 100 and 200 seeds is defined as the final \selfbleu scores. For
% the comparison, \selfbleu scores of \Cklst test cases are calculated
% in the same manner.

\paragraph*{Production rule coverage.} We propose a new metric to evaluate the syntactic diversity of the generated test suite.
It is defined as the number of production rules used in a set of test
sentences. In our experiments, we used the Berkeley Neural
Parser~\cite{kitaev2018seedparser,kitaev2019seedparser} to parse and
collect all the production covered in a set of test sentences.  We
compared the \pdr between 50, 100 and 200 randomly selected \tool
seeds and the \Cklst test cases.
% collect the of 50, 100 and 200 randomly selected \tool seeds and
% \Cklst test cases by the Berkeley Neural
% Parse~\cite{kitaev2018seedparser,kitaev2019seedparser}. Next, we
% traverse the parse trees and the \pdr score is calculated as the
% number of the production rules for the 50, 100 and 200 selected
% \tool seeds and \Cklst test cases.

% ---

In addition, we follow the approach presented by Ma et
al. \cite{ma2018deepgauge}, where the authors measure the coverage of
NLP model intermediate states as corner-case neurons.  Because the
matrix computation of intermediate states impacts NLP model
decision-making, a test suite that covers a greater number of
intermediate states can represent more NLP model decision-making,
making it more diverse.  Specifically, we used two coverage metrics by
Ma et al. \cite{ma2018deepgauge}, \textit{boundary coverage}
(BoundCov) and \textit{strong activation coverage} (SActCov), to
evaluate the test suite diversity.  It is worth noting that a test
sample with a statistical distribution similar to the training data is
rarely found in the corner case region.  Thus, covering a larger
corner case region indicates that the test suite is more likely to be
buggy.


\begin{equation}
\begin{split}
    \text{UpperCorner}(\mathcal{X}) = \{n \in N | \exists x \in \mathcal{X}: f_n(x) \in (high_n, +\infty)\}; \\
    \text{LowerCorner}(\mathcal{X}) = \{n \in N | \exists x \in \mathcal{X}: f_n(x) \in (-\infty, low_n)\}; \\
\end{split}
    \label{eq:corner}
\end{equation}

\noindent Equation \ref{eq:corner} defines the corner-case neuron of
the NLP model $f(\cdot)$, where $\mathcal{X}$ is the given test suite,
$N$ is the number of neurons in model $f(\cdot)$, $f_n(\cdot)$ is the
$n^{th}$ neuron's output, and $high_n$ and $low_n$ are the $n^{th}$
neurons' output bounds on the model training dataset.  Equation
\ref{eq:corner} can be interpreted as the collection of neurons that
emit outputs beyond the model's numerical boundary.

\begin{equation}
\begin{split}
     & BoundCov(\mathcal{X}) = \frac{|UpperCorner(\mathcal{X})| + |LowerCorner(\mathcal{X})| }{2 \times |N|} \\ 
     &\quad  \qquad \qquad  SActCov(\mathcal{X}) = \frac{|UpperCorner(\mathcal{X})|} {|N|} \\ 
\end{split}
    \label{eq:coverage}
\end{equation}

\noindent The definition of our coverage metrics is shown in Equation
\ref{eq:coverage}, where BoundCov measures the coverage of neurons
that produces outputs exceeding the upper or lower bounds, and SActCov
measures the coverage of neurons that creates outputs exceeding the
lower bound.  Higher coverage indicates the test suite is better for
triggering the corner-case neurons, thus better test suite diversity.

% ($Cov(\mathcal{X})$ in \equref{eq:coverage}), where $N$ is the total
% number blocks, $\mathbb I(\cdot)$ is the indicator function, and
% $(B_i(x) > \tau_i))$ represents whether $i^{th}$ block is activated
% by input $x$~(the definition of $B_i$ and $\tau_i$ are the same with
% \equref{eq:new2} and \equref{eq:new3}).  Because AdNNs activate
% different blocks for decision making, then a higher block coverage
% indicates the test samples cover more decision behaviors.


To answer RQ1, for each NLP model under test, we first feed its
training dataset to compute each neuron's lower and upper
bounds. After that, we select the same number of test cases from \tool
and \Cklst as the test suite and compute the corresponding coverage
metrics.

% For each subject, we randomly select 100 seed samples from the test
% dataset as seed inputs. We then feed the same seed inputs into \tool
% and \texttt{ILFO} to generate test samples.  Finally, we feed the
% generated test samples to AdNNs and measure block coverage.  We
% repeat this process 10 times and record the average coverage and the
% variance.  The results are shown in \tabref{tab:coverage} last two
% columns.

%
%The amount of \sa model
%components \sw{?} executed during testing is a critical measurement for
%assessing quality of software testing. A high software coverage
%results in higher chances of unidentified bugs in the \sa model. On
%the other hand, limited distribution only represents narrow portion of
%real world covering limited execution behaviors in a \sa model. It leads
%to detect bugs within the restricted execution behaviors.  Therefore,
%test cases more representative of real-world data result in more generalized
%distribution and higher coverage of the \sa model.
%Therefore, we answer the {\bf RQ2} by
%measuring the neural coverage of the \sa model.
%
%Specifically, we implemented DeepXplore to measure the \sa model
%coverage~\cite{pei2017deepxplore}. \Dxp is the first efficient
%white-box testing framework for large-scale \dl systems. It introduces
%neuron coverage of a set of test inputs as the ratio of the number of
%unique activated neurons and the total number of neurons in input \dl
%system. In this experiment, we compute the neuron coverage of a test
%cases from \tool and \Cklst on the fine-tuned \sw{first time mention fine-tuning: reader does not know how we fine-tune} \sa model of
%\bertsamodel and compare the coverage between \tool and \Cklst.

\paragraph*{\textbf{RQ2 and RQ3}} \jl{need to do the manual study for the \hsd task also?} As described in Section \ref{sec:approach},
\tool generates test cases in two steps: specification-based seed
generation and syntax-based sentence expansion. These automated steps
may generate seed/expanded sentences marked with incorrect sentiment
labels or categorized into wrong linguistic capabilities.
% For example,
% the search rule and template defined in a linguistic capability may
% not always generate seed sentences in that capability or with the
% correct label. 
To answer RQ2 and RQ3, we performed a manual study to measure the
correctness of the sentiment labels and linguistic capabilities
associated with the seed/expanded sentences, produced by \tool.

In the manual study, we randomly sample 100 \tool seed sentences, each
of which has at least one expanded sentence, and divide these seeds to
two sets (i.e., 50 in each set). For each sampled seed sentence, we
randomly obtain one of its expanded sentences.  This forms the two
sets of sentences (200 sentences in total) we use for this study, each
with 50 seeds and 50 corresponding expanded sentences.  We recruited
three participants for this study; all of them are graduate students
with no knowledge about this work.  2 of them were assigned one set of
sentences and the third was assigned the other set.  Each participant
was asked to provide two scores for each sentence.  \emph{(1)
Relevancy score between sentence and its associated \lc}: this score
measures the correctness of \tool linguistic capability
categorization.  The scores are discrete, ranging from 1 (``strongly
not relevant'') to 5 (``strongly relevant''). \emph{(2) sentiment
score of the sentence}: this score measures the sentiment level of the
sentence . It is also discrete, ranging from 1 (``strongly negative'')
to 5 (``strongly positive'').  We measure the following:

\begin{equation}
\begin{aligned}
  Label\_consistency = &\frac{1}{\#Sample}\\
  &\cdot \sum_{i} \delta(label_{S^2LCT}=label_{human})\label{metric:srel}\\
\end{aligned}
\end{equation}
\begin{equation}
\begin{aligned}
  LC\_relevancy_{AVG} = &\frac{1}{\#Sample}\\
  &\cdot\sum_{i} Norm(LC\_relevancy_i) \label{metric:lcrel}
\end{aligned}
\end{equation}

Equation~\ref{metric:srel} represents the percentage of the test cases
that \tool and the participants produce the same sentiment
labels. High value of this metric indicates \tool generates test cases
with correct labels. Equation~\ref{metric:lcrel} represents the
average of the normalized relevancy score between a sentence and its
associated \lc. The relevancy score is to evaluate the relevancy of
the sentence to be used for testing model under test on the
corresponding \lc. Higher average score means indicate the linguistic
capability categorization by \tool is correct. We answer RQ2 and RQ3
using the metrics defined by Equation~\ref{metric:srel} and
Equation~\ref{metric:lcrel}, respectively.





% In particular, we apply the explainable ML techniques to visualize the contribution of each input token to the model predictions.
% 











% \sw{@Simin: add the setup of the bug explanation case study here.}
%In addition to the detection of bugs \sw{have we defined ``bug" in this context?} in the model,
%explanation of the bugs is also important for debugging and
%repairing the model. Therefore, we answer {\bf RQ3} by analyzing the \sa model based on \tool test cases to find the root causes of the bugs.
%
%Specifically, we adapt \Denas{} for this experiment. \sw{Say more what DENAS does.} Rules generated from
%\Denas{} are interpretable functions mapping certain features of the
%input to the expected output of a deep \nn system, and the generated
%rules are considered as the behaviors of the deep \nn system. To
%analyze the bug of a \sa model, we generates rules over the test
%inputs using \Denas{} and identifies faulty rules for the failed test
%inputs.
%\sw{Previous sounds vague. Any specific adaptation of DENAS we did?}
%We manually identify the root of the faulty rules and find the
%root causes of bug in the end.








% \noindent\textbf{Implementation Details.}









% \sw{Missing: environment running these experiments.}

%% \MyPara{Seed Input Selection}
%% %
%% For each linguistic capability, we first search all sentences that
%% meet its requirement. Among found sentences, we randomly select 10
%% sentences due to memory constraint.

%% \MyPara{Word Sentiment}
%% %
%% we extract sentiments of words using the
%% \Swn~\cite{baccianella2010sentiwordnet}. The \Swn is a publicly
%% available lexical resource of words on Wordnet with three numerical
%% scores of objectivity, positivity and negativity. Sentiment word
%% labels from the scores are classified from the algorithm from Mihaela
%% \etal~\cite{mihaela2017sentiwordnetlabel}.

%% \MyPara{\Cfg Expansion}
%% %
%% We build a reference \Cfg of natural language from the English Penn
%% \Trb corpora~\cite{mitchell1993treebank,nltkTreebankCorporaWebPage}.
%% The corpus is sampled from 2,499 stories from a tree year \Wsj
%% collection The \Trb provides a parsed text corpus with annotation of
%% syntactic and semantic structure. In this experiment We implement the
%% \trb corpora available through \Nltk, which is a suite of libraries
%% and programs for \Nlp for English. In addition, we parse the seed
%% input using into its CFG using the Berkeley Neural
%% Parser~\cite{kitaev2018constituency, kitaev2019multilingual}, a
%% high-accuracy parser with models for 11 languages. The input is a raw
%% text in natural language and the output is the string representation of
%% parse tree. Next after comparing CFGs between reference and seed input,
%% we randomly select 10 expansions for generating templates due to
%% memory constraint.

%% \MyPara{Synonyms}
%% %
%% \Model searches synonyms of each token from synonym sets extracted
%% from \Wrdnt using \Spacy open-source library for NLP.

%% \MyPara{Models}
%% %
%% We evaluate the following \sa models via \Model:
%% \Bert~\cite{devlin2019bert}, \Roberta~\cite{liu2019roberta} and
%% \Dbert~\cite{sanh2019distilbert}. These models are fine-tuned on \Sstt
%% and their accuracies are \BertAcc, \RobertaAcc and \DbertAcc.

%% \MyPara{Retraining}
%% %
%% We retrain \sa models. we split \Model generated test cases into
%% train/validation/test sets with the ratio of 8:1:1. The number of
%% epochs and batch size for retraining are 1 and 16 respectively.




%
\begin{figure}
    \centering
    \includegraphics[width=0.5\textwidth]{figs/covergae.pdf}
    \caption{The coverage results of the generated test
      samples. \sw{Make the figures larger.}}
    \label{fig:coverage}
\end{figure}


Fig. \ref{fig:coverage} shows the coverage results of the generated
test samples, where the red line represents \tool and the black line
represents \Cklst.  Each column in Fig. \ref{fig:coverage} represent
the results for one NLP model, the first row is the \textit{BoundCov}
results and the second row is the \textit{SActCov} results.  From the
results, we make two observations observations. First, for \emph{all}
experimental settings (\eg NLP model, coverage metric), \tool achieves
high coverage than \Cklst. Recall that a higher coverage implies the
test case in the test suite is more diverse and rarely to have a
statistical distribution similar to the model training data. As a
result, a test suite with greater coverage complements the model
training data distribution (\ie hold-out testing data) better.  The
experimental results confirm that \tool can generate more diverse test
cases to complement the hold-out testing data for testing NLP models.
\sw{What does the growth trend in each figure indicate?} \sw{Does the
  absolute numbers or relative difference of the two lines on y axis
  mean anything concrete? E.g., how significant is the improvement in
  diversity?}

Another interesting finding is that for each NLP model, there is no
fixed relationship between \textit{BoundCov} and \textit{SActCov}. In
other words, while a test suite may produce higher \textit{BoundCov}
for some models, the same test suite may get higher \textit{SActCov}
for other NLP models.  Recall that \textit{BoundCov} measures both the
upper and lower corner neurons and \textit{SActCov} measures only the
upper corner neurons.  Such observation implies that the upper and
lower corner neurons are distributed unevenly, and measuring only one
of them is not enough.


%\input{CM/explain}
